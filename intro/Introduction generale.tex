\begin{introductiongenerale}

    \vspace{.5cm}

    Dans le contexte d’un nombre croissant d’efforts visant à protéger la vie privée au Maroc, la commission nationale de contrôle de la protection des données à caractère personnel (CNDP) a été créée par la loi n°09-08 du 18 février 2009. Cette commission est chargée de vérifier que les traitements des données personnelles sont licites, légaux et qu’ils ne portent pas atteinte à la vie privée, aux libertés et droits fondamentaux de l’homme. 
    \newline

    \noindent Conformément aux exigences de la CNDP, la Banque Centrale Populaire est tenue de maintenir un registre du traitement détaillant l’ensemble des activités de traitement effectuées par la banque. Chaque traitement doit faire l’objet d’une fiche décrivant un sous-ensemble des caractéristiques du traitement en question. Ces caractéristiques incluent les catégories de données traitées, les finalités du traitement, les personnes concernées, les accès et les communications des données, les transferts des données vers l’étranger, ainsi que plusieurs autres.
    \newline

    \noindent Actuellement, le registre du traitement de la banque existe sous forme de fichiers Excel et donc présente une certaine difficulté en matière de revue, de consultation, et de modification. Suite à ces contraintes, la fonction Conformité Groupe pour la BCP a exprimé un besoin d’automatisation du registre du traitement. Le présent travail constitue donc une description structurée de l’ensemble des efforts entrepris afin de répondre à ce besoin. 
    
\end{introductiongenerale}